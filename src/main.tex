%! Author = jonathan
%! Date = 2/22/24

% Packages
\documentclass{sig-alternate-per}
\begin{document}
    \title{Alternate {\ttlit ACM} SIG Proceedings Paper in LaTeX
    Format}

    \numberofauthors{2}

    \author{
        % First Author
        \alignauthor
        Ben Trovato\titlenote{Dr.~Trovato insisted his name be first.}\\
        \affaddr{Institute for Clarity in Documentation}\\
        \email{trovato@corporation.com}
        % Second Author
        \alignauthor
        G.K.M. Tobin\titlenote{The secretary disavows
        any knowledge of this author's actions.}\\
        \affaddr{Institute for Clarity in Documentation}\\
        \email{webmaster@marysville-ohio.com}
    }

    \maketitle
    \begin{abstract}
        This paper provides a sample of a \LaTeX\ document which conforms,
        somewhat loosely, to the formatting guidelines for
        ACM SIG Proceedings.
        It is an {\em alternate} style which produces
        a {\em tighter-looking} paper and was designed in response to
        concerns expressed, by authors, over page-budgets.
        It complements the document \textit{Author's (Alternate) Guide to
        Preparing ACM SIG Proceedings Using \LaTeX$2_\epsilon$\ and Bib\TeX}.
        This source file has been written with the intention of being
        compiled under \LaTeX$2_\epsilon$\ and BibTeX.

        The developers have tried to include every imaginable sort
        of ``bells and whistles", such as a subtitle, footnotes on
        title, subtitle and authors, as well as in the text, and
        every optional component (e.g. Acknowledgments, Additional
        Authors, Appendices), not to mention examples of
        equations, theorems, tables and figures.

        To make best use of this sample document, run it through \LaTeX\
        and BibTeX, and compare this source code with the printed
        output produced by the dvi file.
        A compiled PDF version
        is available on the web page to help you with the
        `look and feel'.
    \end{abstract}
%    %! Date = 2/22/24

\begin{abstract}
    We propose \emph{Aristos}~\footnote{Translation of \textbf{optimal} in Greek}
    a communication-optimal, distributed algorithm that uses \emph{asynchronous communication}
    interleaved with computation for specifically tackling the communication overhead of
    Distributed Mixture-of-Experts (DMoE) transformer models.
    DMoE as implemented today entails frequent synchronous~\verb|All-to-All| communication operations
    that scale linearly~\emph{per step} with a model's number of layers.
    Thus, we first seek clarification on how~\verb|All-to-All| bottlenecks DMoE
    and whether the interconnect or communication algorithm is at fault.

    We investigate more than \textbf{21k}~\verb|All-to-All| CUDA kernels
    and empirically confirm that their runtimes exhibit a long-tail distribution
    with worst-case slow-down of
    \textbf{8.9}X and \textbf{100}X in multi-node and single-node settings, respectively.
    We argue that this phenomenon is a shortcoming of the \emph{global barrier}
    necessitated by the synchronous implementation of~\verb|All-to-All| in state of the art
    collective libraries such as NCCL\@.
    We use these empirical insights to motivate Aristos, which obviates the global barrier,
    instead favoring asynchronous communication yielding native support for pipelining.
    Aristos also exposes tunable hyperparameters that navigate the tradeoff of faster performance and
    reduced token dropping.
\end{abstract}

    \section{Introduction}\label{sec:introduction}
    The \textit{proceedings} are the records~\cite{Lamport:LaTeX} of a conference.
    ACM seeks to give these conference by-products a uniform,
    high-quality appearance.  To do this, ACM has some rigid
    requirements for the format of the proceedings documents: there
    is a specified format (balanced  double columns), a specified
    set of fonts (Arial or Helvetica and Times Roman) in
    certain specified sizes (for instance, 9 point for body copy),
    a specified live area (18 $\times$ 23.5 cm [7" $\times$ 9.25"]) centered on
    the page, specified size of margins (1.9 cm [0.75"]) top, (2.54 cm [1"]) bottom
    and (1.9 cm [.75"]) left and right; specified column width
    (8.45 cm [3.33"]) and gutter size (.83 cm [.33"]).

    The good news is, with only a handful of manual
    settings, the \LaTeX\ document
    class file handles all of this for you.

    The remainder of this document is concerned with showing, in
    the context of an ``actual'' document, the \LaTeX\ commands
    specifically available for denoting the structure of a
    proceedings paper, rather than with giving rigorous descriptions
    or explanations of such commands.
%    %! Author = jonathan
%! Date = 2/22/24

\section{Introduction}\label{sec:introduction}
TODO
%%
%%    %! Author = jonathan
%! Date = 2/22/24

\section{Acknowledgments}\label{sec:acknowledgments}
TODO
%
    \bibliographystyle{abbrv}
    \bibliography{biblio}

\end{document}
