%! Author = jonathan
%! Date = 2/22/24

% Packages
\documentclass{sig-alternate-per}
\begin{document}
    \title{Alternate {\ttlit ACM} SIG Proceedings Paper in LaTeX
    Format}

    \numberofauthors{2}

    \author{
        % First Author
        \alignauthor
        Ben Trovato\titlenote{Dr.~Trovato insisted his name be first.}\\
        \affaddr{Institute for Clarity in Documentation}\\
        \email{trovato@corporation.com}
        % Second Author
        \alignauthor
        G.K.M. Tobin\titlenote{The secretary disavows
        any knowledge of this author's actions.}\\
        \affaddr{Institute for Clarity in Documentation}\\
        \email{webmaster@marysville-ohio.com}
    }

    \maketitle
    \begin{abstract}
        This paper provides a sample of a \LaTeX\ document which conforms,
        somewhat loosely, to the formatting guidelines for
        ACM SIG Proceedings.
        It is an {\em alternate} style which produces
        a {\em tighter-looking} paper and was designed in response to
        concerns expressed, by authors, over page-budgets.
        It complements the document \textit{Author's (Alternate) Guide to
        Preparing ACM SIG Proceedings Using \LaTeX$2_\epsilon$\ and Bib\TeX}.
        This source file has been written with the intention of being
        compiled under \LaTeX$2_\epsilon$\ and BibTeX.

        The developers have tried to include every imaginable sort
        of ``bells and whistles", such as a subtitle, footnotes on
        title, subtitle and authors, as well as in the text, and
        every optional component (e.g. Acknowledgments, Additional
        Authors, Appendices), not to mention examples of
        equations, theorems, tables and figures.

        To make best use of this sample document, run it through \LaTeX\
        and BibTeX, and compare this source code with the printed
        output produced by the dvi file.
        A compiled PDF version
        is available on the web page to help you with the
        `look and feel'.
    \end{abstract}
%    %! Date = 2/22/24

\begin{abstract}
    Through sparsely activated computation, the Mixture-of-Experts (MoE) architecture
    mitigates the skyrocketing scaling costs and power consumption of larger Deep Learning models.
    Existing work demonstrates that this architecture achieves these savings without compromising
    latency or model accuracy.
    However, distributed MoE computation as implemented today necessitates frequent \emph{synchronous}~\ata
    communication that poses significant overhead, especially at scale.
    We seek to clarify and address synchronous All-to-All bottlenecks in distributed MoE computation
    both in single-node and multi-node settings.

    Precisely, we investigate more than \textbf{23k}~\ata CUDA kernels and empirically confirm that
    their runtimes exhibit a long-tail distribution
    with an average and a worst-case slow-down of \textbf{4X} and \textbf{50X}, respectively.
    We argue that this phenomenon is a shortcoming of the \emph{global synchronization}
    necessitated by the synchronous implementation of~\ata in state of the art collective libraries such as NCCL\@.
    Second, we further show that even within a high-bandwidth supercomputing cluster,
    inter-node~\ata communication between two nodes is about \textbf{7X} slower compared
    to a single-node with equal number of GPUs.
    As such, we propose a novel algorithm that uses \emph{asynchronous communication}
    interleaved with compute and we argue for expert stacking and topology-awareness in
    automatic expert parallelism.
\end{abstract}

    \section{Introduction}\label{sec:introduction}
    The \textit{proceedings} are the records~\cite{Lamport:LaTeX} of a conference.
    ACM seeks to give these conference by-products a uniform,
    high-quality appearance.  To do this, ACM has some rigid
    requirements for the format of the proceedings documents: there
    is a specified format (balanced  double columns), a specified
    set of fonts (Arial or Helvetica and Times Roman) in
    certain specified sizes (for instance, 9 point for body copy),
    a specified live area (18 $\times$ 23.5 cm [7" $\times$ 9.25"]) centered on
    the page, specified size of margins (1.9 cm [0.75"]) top, (2.54 cm [1"]) bottom
    and (1.9 cm [.75"]) left and right; specified column width
    (8.45 cm [3.33"]) and gutter size (.83 cm [.33"]).

    The good news is, with only a handful of manual
    settings, the \LaTeX\ document
    class file handles all of this for you.

    The remainder of this document is concerned with showing, in
    the context of an ``actual'' document, the \LaTeX\ commands
    specifically available for denoting the structure of a
    proceedings paper, rather than with giving rigorous descriptions
    or explanations of such commands.
%    %! Date = 2/22/24

\section{Introduction}\label{sec:introduction}
Current work shows the Mixture-of-Experts (MoE) architecture~\cite{10.1162/neco.1991.3.1.79}
is a productive answer to the problem of efficiently training and serving larger models
up to trillions of parameters~\cite{DBLP:journals/corr/abs-2101-03961} and facilitating sequence length in the order of
millions~\cite{Gemini_Team_2024}.
In current implementations, the MoE architecture entails substituting the Feed-Forward Network (FFN) of a transformer
encoder or decoder with $n$ replicas termed \emph{experts} and a gating network, collectively comprising an MoE layer
~\cite{ShazeerMMDLHD17}.
The key insight is ~\emph{conditional computation}, which entails sparse utilization
of a subset of a model’s parameters contingent on its input~\cite{doi:10.1142/S0218001403002411}.
Empirical work from the DeepSpeed team of Microsoft shows 5x less training time for a GPT-3 MoE model at 1.3B and
7.3X faster inference for a trillion-parameter MoE model~\cite{pmlr-v162-rajbhandari22a}.
Notably, recently released Gemini 1.5, the first multimodal model to support context length of millions of tokens,
adopts this architecture~\cite{Gemini_Team_2024}.

On the other hand, this architecture introduces new algorithmic and systems challenges,
such as load balancing across experts~\cite{ShazeerMMDLHD17}, training instability~\cite{NEURIPS2022_3e67e84a},
expert capacity restriction leading to token dropping~\cite{gale2022megablocks},
and collective communication overhead~\cite{DBLP:journals/corr/abs-2006-16668}.
In a distributed setting
%%
%%    %! Author = jonathan
%! Date = 2/22/24

\section{Acknowledgments}\label{sec:acknowledgments}
This work is supervised by Dr. Rachee Singh, whom we wish to express gratitude for her guidance.
We acknowledge that this research used supercomputing resources of the National Energy Research
Scientific Computing Center (NERSC), a Department of Energy Office of Science User Facility
supported under Contract No. DE-AC02-05CH11231, using NERSC award NERSC DDR-ERCAP0027296.
We immensely thank Dr. Guila Guidi for introducing the authors to NERSC Perlmutter and
are grateful to Dr. Erik Palmer of LBNL for his efforts elucidating Perlmutter's network topology.
%
    \bibliographystyle{abbrv}
    \bibliography{biblio}

\end{document}
