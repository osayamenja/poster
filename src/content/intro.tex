%! Date = 2/22/24

\section{Introduction}\label{sec:introduction}
Since the advent of the Transformer~\cite{NEURIPS2017_3f5ee243}, existing trends in language model complexity
indicate a steep exponential increase~\cite{DBLP:journals/corr/abs-2201-11990} in model parameters.
These larger Transformer models achieve significantly better performance
~\cite{DBLP:journals/corr/abs-2005-14165, DBLP:journals/corr/abs-2001-08361},
hence the appeal to scale their size continuously.
However, this scale demands exorbitant compute resources,
necessitating a distributed setting as memory requirements exceed the capacity of any
single accelerator (GPU, TPU, FPGA, etc.)~\cite{DBLP:journals/corr/abs-2201-11990}
This interplay begs the question:
\emph{how do we sidestep the high energy and hardware costs of scaling Large Transformer models?}

Current work shows the Mixture-of-Experts (MoE) architecture~\cite{10.1162/neco.1991.3.1.79}
is a productive answer to this conundrum.
This architecture entails substituting the Feed-Forward Network (FFN) of a transformer
encoder or decoder with $n$ replicas termed \emph{experts} and a gating network, collectively comprising an MoE layer
~\cite{ShazeerMMDLHD17}.
The key insight is ~\emph{conditional computation}: sparse utilization
of a subset of a model’s parameters contingent on its input~\cite{doi:10.1142/S0218001403002411}.
Empirical work from the DeepSpeed team of Microsoft shows 5x less training time for a GPT-3 MoE model at 1.3B and
7.3X faster inference for a trillion-parameter MoE model~\cite{pmlr-v162-rajbhandari22a}.
Notably, recently released Gemini 1.5, the first multimodal model to support context length of millions of tokens,
adopts this architecture~\cite{Gemini_Team_2024}.

On the other hand, this architecture introduces new algorithmic and systems challenges,
such as load balancing across experts~\cite{ShazeerMMDLHD17}, training instability~\cite{NEURIPS2022_3e67e84a},
expert capacity restriction leading to token dropping~\cite{gale2022megablocks},
and collective communication overhead~\cite{DBLP:journals/corr/abs-2006-16668}.